\documentclass{scrartcl}

\usepackage{amsmath}
\usepackage{amssymb}
\usepackage{oz}
\usepackage{graphicx}
\usepackage{subcaption}
\usepackage{multirow}
\usepackage{listings}
\usepackage{csquotes}
\usepackage{enumerate}
\usepackage{scrextend}
\usepackage{sidenotes}

\title{Excerpt of: Verification Condition Splitting}

\begin{document}
\begin{center}
    \Large{\textbf{Excerpt of}}

    \LARGE{Verification Condition Splitting}

    \large{by K. Rustan M. Leino , Michał Moskal, and Wolfram Schulte}
\end{center}

\vspace{1cm}

\begin{addmargin}[0.2\linewidth]{0.2\linewidth}
    \begin{center}
        \textbf{Key questions}
    \end{center}
    \begin{enumerate}[i]
        \item What do Dafnys options regarding \enquote{Verification-condition splitting} do?
            \label{cli}
        \item How does Dafny split verification conditions? \label{split}
        \item How does the approach compare to our \enquote{cut-points} approach?
            \label{cut-points}
        \item Can Dafnys verification condition splitting be used as an approach to\
            diversify a parallel portfolio that solves .dfy files?
    \end{enumerate}
\end{addmargin}

\vspace{1cm}

The paper explains the traditional approach of VCG and goes on to mention
\enquote{modular verification}, that is generating a VC for each part of the program.

The basic idea is to \enquote{split VCs using control-flow information}.
The splitting is done \enquote{at the level of the intermediate language}.
Multiple proof obligations in the same branch are kept together,
as lemmas discovered during the proof of one obligation may be useful for subsequent ones.

The authors carefully define a type of CFGs that has \enquote{assume} and \enquote{assert}.
These CFGs are acyclic, because the loops are replaced by an induction
(it is assumed that the invariant is known).
Furthermore, the vertices have no more than two children
(can be accomplished through transofmation).
\textit{An example can be found in figure 0 (\enquote{A horizontal split}).
Note that there presubambly is an error: the third vertex should have \(\all\)
instead of \(\exists\). The vertex is the induction hypothesis for any \(i_1\).}


They define that a CFG \enquote{goes wrong} if an execution trace exists
such that all assumptions on it evaluate to true but the last node of the trace has an assertion
that evaluates to false.
\textit{Assuming that all asserts of a \enquote{program part} are combined to one VC
(before splitting),}
there is a one-to-one relation between verifcation conditions and CFGs,
denoted by \(wrong(\cdot)\) in the paper.
\textit{This assumption is justified as }
\enquote{lemmas discovered during the proof of one proof obligation [might] turn out to be useful    
also for other proof obligations in the same branch of the program.}
\sidenote{q. \ref{split}}
They go on to define horizontal (at a branching) and vertical
(by removing some assertions for one CFG and the remaining ones for the other) splits.
Correctness is proven.
\textit{In Boogies help message horizontal splits are referred to as \enquote{path splitting},
    vertical splits as \enquote{assertion splitting}.}


A vertical split is performed if no horizontal split is performed.
This is determined by a cost heuristic:
A particular horizontal split into CFGs \(G_0\) and \(G_1\) is chosen if 
\[
    \mathcal{K}_\mathcal{G}(G)^2 \le \gamma_H
    \cdot (\mathcal{K}_\mathcal{G}(G_0)^2 + \mathcal{K}_\mathcal{G}(G_1)^2)
\]
where \(\mathcal{K}_\mathcal{G}(G)\) is the cost of a CFG \(G\).
\textit{The paper, most likely, got the inequality sign in
    \(\mathcal{K}_\mathcal{G}(G)^2 \le \gamma_H
    \cdot (\mathcal{K}_\mathcal{G}(G_0)^2 + \mathcal{K}_\mathcal{G}(G_1)^2)\).
    wrong. It should be
    \(\mathcal{K}_\mathcal{G}(G)^2 \ge \gamma_H
    \cdot (\mathcal{K}_\mathcal{G}(G_0)^2 + \mathcal{K}_\mathcal{G}(G_1)^2)\).
    This is fixed in Boogies help message.
}

The paper defines the cost heuristic for formulas \(\psi\),
labels \(l\) (\(l = assert\ \psi\) or \(l = assume\ \psi\)),
nodes \(v\) and CFGs \(G\) as

\[\mathcal{K}_\mathcal{G}(G) = \sum_{v\in V} \mathcal{K}_V(v)\]
        
\[\mathcal{K}_V(n) = \mathcal{K}_\mathcal{L}(L(n)) \cdot \gamma_V + \mathcal{K}_E(n))\]
    
\[\mathcal{K}_E(n) =
    \begin{cases}
        1, \text{when \(n\) is the entry point}\\
        \mathcal{K}_E(n'), \text{when \(n\) has indegree 1 and \((n', n) \in E\)}\\
        \gamma_E \cdot \sum_{(n', n) \in E}, \text{otherwise}
    \end{cases}
\]

\[
    \mathcal{K}_\mathcal{L}(assume\ \psi) =
    \gamma_\mathcal{L} \cdot \mathcal{K}_\mathcal{T}(\psi)\\
    \mathcal{K}_\mathcal{L}(assert\ \psi) = \mathcal{K}_\mathcal{T}(\psi)
\]

In the implementation of the paper \(\mathcal{K}_\mathcal{T}(\psi)\) is constant.
\textit{Boogies help message indicates that \(\mathcal{K}_\mathcal{T}(\psi) = 1\).}


\sidenote{q. \ref{cut-points}}
On first sight the horizontal splits align perfectly with what we planned for our
\enquote{cut-point} approach.
The difference is: splitting of the CFG vs splitting of the VC.
It is to be understood whether that makes functional difference.


The paper suggests two modes of operation: static and dynamic spitting.
Static splitting repeatedly splits the VC with the largest cost
until splitting is no longer possible or a limit is reached.

Dynamic splitting suggests embedding the splitting procedure into a prover in the following way:
Try to prove the VC with a timeout of \(t_1\).
If the time runs out, split into \(k\) pieces (if possible).
Repeat this for each (partial) VC until the split is not possible
(i.e. there is only one assert on a single path)
and run the prover on that VC with a timeout of \(t_2\).

\textit{The paper refers to both \(t_1\) and \(t_2\) as \(t\) which is rejectable.}

\textit{Boogie implements an an initial split. This is not discussed in the paper.}

\textit{The dynamic mode is called \enquote{keep going mode} in the Boogie help message.}
    

\sidenote{q. \ref{cli}}
The paper does not refer to dafnys or boogies command line options.
The unclear relation between \enquote{verification condition splitting}
and \enquote{CFG splitting} specifically creates further confusion.
\textit{As the paper does not describe a way to split particular verifaction conditions
without splitting the entire CFG (one CFG may contain more than one VCs)}

\textit{Despite there not being a reliable mapping of the coefficients and thresholds in the paper
to the parameters,
I presume the follwing.}

\begin{tabular}{|l|l|}
    \hline
    CLI Parameter & Equivalent in the paper \\
    \hline
    /vcsMaxCost:\textlangle f \textrangle
                  & The minimum value for \(\mathcal{K}_\mathcal{V}(n)^2\), to be eligable for a split\\
    /vcsMaxSplits: \textlangle n \textrangle & - \textdagger \\
    /vcsMaxKeepGoingSplits: \textlangle n \textrangle  & \(k\) \\
    /vcsKeepGoingTimeout: \textlangle n \textrangle  & \(t_1\) \\
    /vcsFinalAssertTimeout: \textlangle n \textrangle  & \(t_2\) \\
    /vcsPathJoinMult: \textlangle f \textrangle  & \(\gamma_E\) \\
    /vcsPathCostMult: \textlangle f1 \textrangle  & related to \(\gamma_V\) * \\
    /vcsAssumeMult: \textlangle f2 \textrangle  & \(\gamma_\mathcal{L}\) \\
    /vcsPathSplitMult: \textlangle f \textrangle  & \(\gamma_H\) \\
    /vcsDumpSplits & - \\
    /vcsCores: \textlangle n \textrangle  & - \\
    /vcsLoad: \textlangle f \textrangle  & - \\
    \hline
\end{tabular}

\begin{footnotesize}
    * In Boogie the definition of \(\mathcal{K}_V(n)\) is
    \(\mathcal{K}_V(n) = \mathcal{K}_\mathcal{L}(L(n)) \cdot (1 + \text{f1} \cdot \mathcal{K}_E(n))\).\\
    \textdagger The paper does not suggest an initial split of the VC as is implemented in Boogie.
\end{footnotesize}

\end{document}
