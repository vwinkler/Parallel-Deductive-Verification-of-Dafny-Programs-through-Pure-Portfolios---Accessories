\documentclass{scrartcl}

\usepackage{amsmath}
\usepackage{amssymb}
\usepackage{oz}
\usepackage{graphicx}
\usepackage{subcaption}
\usepackage{multirow}
\usepackage{listings}
\usepackage{csquotes}
\usepackage{enumerate}
\usepackage{scrextend}

\title{Excerpt of: Verification Condition Splitting}

\begin{document}
\begin{center}
    \Large{\textbf{Excerpt of}}

    \LARGE{Verification Condition Splitting}

    \large{by K. Rustan M. Leino , Michał Moskal, and Wolfram Schulte}
\end{center}

\vspace{1cm}

\begin{addmargin}[0.2\linewidth]{0.2\linewidth}
    \begin{center}
        \textbf{Key questions}
    \end{center}
    \begin{enumerate}[i]
        \item What do Dafnys options regarding \enquote{Verification-condition splitting} do?
        \item How does Dafny split verification conditions?
        \item How does the approach compare to our \enquote{cut-points} approach?
        \item Can Dafnys verification condition splitting be used as an approach to\
            diversify a parallel portfolio that solves .dfy files?
    \end{enumerate}
\end{addmargin}

\vspace{1cm}

The paper explains the traditional approach of VCG and goes on to mention
\enquote{modular verification}, that is generating a VC for each part of the program.

The basic idea is to \enquote{split VCs using control-flow information}.
The splitting is done \enquote{at the level of the intermediate language}.
Multiple proof obligations in the same branch are kept together,
as lemmas discovered during the proof of one obligation may be useful for subsequent ones.

The authors carefully define a type of CFGs that has \enquote{assume} and \enquote{assert}.
They define that a CFG \enquote{goes wrong} if an execution trace exists
such that all assumptions on it evaluate to true but the last node of the trace has an assertion
that evaluates to false.
They go on to define horizontal (at a branching) and vertical
(by removing some assertions for one CFG and the remaining ones for the other) splits.
Correctness is proven.

On first sight the horizontal splits align perfectly with what we planned for our
\enquote{cut-point} approach.
The difference is: splitting of the CFG vs splitting of the VC.
It is to be understood whether that makes functional difference.

\end{document}
