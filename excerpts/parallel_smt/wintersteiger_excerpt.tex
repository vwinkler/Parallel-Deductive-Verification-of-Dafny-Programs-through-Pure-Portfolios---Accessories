\documentclass{scrartcl}

\usepackage{amsmath}
\usepackage{amssymb}
\usepackage{oz}
\usepackage{graphicx}
\usepackage{subcaption}
\usepackage{multirow}
\usepackage{listings}
\usepackage{csquotes}
\usepackage{enumerate}
\usepackage{scrextend}

\title{Excerpt of A Concurrent Portfolio Approach to SMT Solving}

\begin{document}
\begin{center}
    \Large{\textbf{Excerpt of}}

    \LARGE{A Concurrent Portfolio Approach to SMT Solving}

    \large{by Christoph M. Wintersteiger, Youssef Hamadi, and Leonardo de Moura}
\end{center}

\vspace{1cm}

\begin{addmargin}[0.2\linewidth]{0.2\linewidth}
    \begin{center}
        \textbf{Key questions}
    \end{center}
    \begin{enumerate}[i]
        \item What is the state of the art in parallel SMT solving?
        \item How successfull are parallel SMT solvers?
        \item Are they used particularly in software verification?
        \item Have (parallel) porfolios been used in SMT solving?
        \item How is diversification achieved?
    \end{enumerate}
\end{addmargin}

\vspace{1cm}

Wintersteiger, Hamadi, and de Moura describe their parallel version of Z3.
They apply the parallel portfolio approach by running four sequential Z3 solvers
with different configurations.

Their benchmarks are from the QF\_IDL (quantifier-free integer difference logic) category
of the SMT library.
They tested four different portfolios that yielded speedups \(S\).
Many benchmarks took sequential Z3 only less than \(60\, s\) to solve
hence the parallelization overhead dominates.
Excluding those benchmarks resulted in a much higher speedup of \(S'\).
\begin{itemize}
    \item four arithmetic solvers without sharing \(\,\to\, S < 1.0\)
    \item four arithmetic solvers (with sharing) \(\,\to\, S = 1.06, S' = 3.2\)
    \item four solvers with diversified SAT strategies (with sharing)
        \(\,\to\, S = 1.14, S' = 2.6\)
    \item four solvers with randomized seeds (with sharing)\(\,\to\, S = 1.28, S' = 3.5\)
\end{itemize}

\textit{I believe QF\_IDL is, as a subset of ILP, decidable and in \(NP\)
I wonder if the portfolio approach would have performed better on more difficult
benchmarks.}

The paper is very vague and ambiguous when it comes to its diversification approach.
To quote the full description:
\enquote{The default configuration of our implementation uses the same portfolio as
ManySAT on the first four cores, i.e., it only diversifies based on heuristics spe-
cific to the built-in SAT-solver. Additional cores are configured by using different
combinations of settings. Other configurations may be set from the command
line, a feature available to the user in order to define their own portfolios.}
In the final section of the paper, the authors expect better portfolios
from the practical application of their solver.

Furthermore, it looks like the four arithmetic solvers are:
a QF\_UF (quantifier-free uninterpreted functions logic) solver,
two algorithmically different difference-logic solvers,
and one simplex solver.
The first three are over-approximations that can only definitively determine unsatisfiability.
These solvers are mentioned in the \enquote{Portfolios} section of the paper.

As a side note:
The paper claims that
\enquote{most benchmarks in the SMT library require a relatively small amount of memory to solve}
and that \enquote{it is rare to see benchmarks that require more than 100 MB}.


\end{document}
