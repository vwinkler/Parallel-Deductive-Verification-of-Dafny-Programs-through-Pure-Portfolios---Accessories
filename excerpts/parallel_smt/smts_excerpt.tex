\documentclass{scrartcl}

\usepackage{amsmath}
\usepackage{amssymb}
\usepackage{oz}
\usepackage{graphicx}
\usepackage{subcaption}
\usepackage{multirow}
\usepackage{listings}
\usepackage{csquotes}
\usepackage{enumerate}
\usepackage{scrextend}

\title{Excerpt of The OpenSMT Solver in SMT-COMP 2021}

\begin{document}
\begin{center}
    \Large{\textbf{Excerpt of}}

    \LARGE{The OpenSMT Solver in SMT-COMP 2021}

    \large{by Masoud Asadzade, Martin Blicha, Antti E. J. Hyvarinen, and Natasha Sharygina}
\end{center}

\vspace{1cm}

\begin{addmargin}[0.2\linewidth]{0.2\linewidth}
    \begin{center}
        \textbf{Key questions}
    \end{center}
    \begin{enumerate}[i]
        \item What is the state of the art in parallel SMT solving?
        \item How successfull are parallel SMT solvers?
        \item Have (parallel) porfolios been used in SMT solving?
    \end{enumerate}
\end{addmargin}

\vspace{1cm}

There exist two different versions.
One randomises the SAT solver \enquote{choosing \(2\, \%\) of the decision variables randomly}
and the other applies the cube-and-conquer paradigm.

\end{document}
