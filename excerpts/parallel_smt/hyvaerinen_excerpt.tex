\documentclass{scrartcl}

\usepackage{amsmath}
\usepackage{amssymb}
\usepackage{oz}
\usepackage{graphicx}
\usepackage{subcaption}
\usepackage{multirow}
\usepackage{listings}
\usepackage{csquotes}
\usepackage{enumerate}
\usepackage{scrextend}

\title{Excerpt of Parallel Satisfiability Modulo Theories}

\begin{document}
\begin{center}
    \Large{\textbf{Excerpt of}}

    \LARGE{Parallel Satisfiability Modulo Theories}

    \large{by Antti E.J. Hyvärinen and Christoph M. Wintersteiger}
\end{center}

\vspace{1cm}

\begin{addmargin}[0.2\linewidth]{0.2\linewidth}
    \begin{center}
        \textbf{Key questions}
    \end{center}
    \begin{enumerate}[i]
        \item What is the state of the art in parallel SMT solving?
        \item How successfull are parallel SMT solvers?
        \item Are they used particularly in software verification?
        \item Have (parallel) porfolios been used in SMT solving?
        \item How is diversification achieved?
    \end{enumerate}
\end{addmargin}

\vspace{1cm}

SMT solving is mentioned as the \enquote{traditional application field of SMT solvers}.
It is stated that SMT solving exhibits an inherent randomness
that lends itself to a parallel portfolio approach.
Previous work on SMT portfolios can be found in
\enquote{A Concurrent Portfolio Approach to SMT Solving} by Wintersteiger
which is reported to compare \enquote{favorably to many other portfolio-based approaches}.
The paper only very briefly strikes diversification techniques
and mentions several examples of places for randomization without giving details:
\begin{itemize}
    \item \enquote{explanation generation inside the theory solvers}
    \item \enquote{the process used for selecting decision literals}
    \item tie breaking
    \item \enquote{a heuristic equivalence parameter}
    \item \enquote{mixing random heuristics together with a more context-dependent heuristic}
    \item \enquote{different algorithms for the core theory solvers}
\end{itemize}


Figure 5.1 is an interesting presentation.

The paper exclaims:
\enquote{Sharing of learned lemmas plays a central role in parallel SMT.}
About three pages including some pseudocode is dedicated to this aspect.

The rest of the paper (about ten pages) highlights SMT solver parallelization strategies
other than portfolios.
These are search-space partitioning and decomposition,
A partitioning of a formula is an equivalent disjunction into formulas
whose pairwise conjunctions are unsatisfiable.
This entails that searching for a satisfying assignment at the same \enquote{coordinates}
is futile as one should expect from a partitioning.
Decomposition is introduced as an alternative to partitioning,
without this feature.

As a side note: The paper has an interesting way to present the rationale of a portfolio approach.
The runtime of a SMT solver (for a particular formula) is understood as a random variable \(T\).
Figure 5.1 displays the probability densitiy function of \(T\) for two different formulas.
The paper acknowledges:
\enquote{the difference between the two distributions means that the instances will benefit
from parallelization approaches in very different ways}.

\end{document}
