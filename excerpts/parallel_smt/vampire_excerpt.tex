\documentclass{scrartcl}

\usepackage{amsmath}
\usepackage{amssymb}
\usepackage{oz}
\usepackage{graphicx}
\usepackage{subcaption}
\usepackage{multirow}
\usepackage{listings}
\usepackage{csquotes}
\usepackage{enumerate}
\usepackage{scrextend}

\title{Excerpt of Vampire 4.6-SMT System Description}

\begin{document}
\begin{center}
    \Large{\textbf{Excerpt of}}

    \LARGE{Vampire 4.6-SMT System Description}

    \large{by Giles Reger, Martin Suda, Andrei Voronkov, Laura Kovacs,
        Evgeny Kotelnikov, Simon Robillard, Martin Riener, Michael Rawson,
        Bernhard Gleiss, Jakob Rath, Petra Hozzova, Johannes Schoisswohl,
        Ahmed Bhayat, Petra Hozzova, and Marton Hajdu
    }
\end{center}

\vspace{1cm}

\begin{addmargin}[0.2\linewidth]{0.2\linewidth}
    \begin{center}
        \textbf{Key questions}
    \end{center}
    \begin{enumerate}[i]
        \item What is the state of the art in parallel SMT solving?
        \item How successfull are parallel SMT solvers?
        \item Are they used particularly in software verification?
        \item Have (parallel) porfolios been used in SMT solving?
    \end{enumerate}
\end{addmargin}

\vspace{1cm}

\enquote{Vampire is a parallel portfolio solver, executing a schedule of complementary
strategies in parallel.}

There is very little information on their parallelization approach except:
\enquote{In the parallel track Vampire will run using its parallel portfolio mode}.

Additional information:
On the website (vprover.github.io/usage.html) they call the portfolio mode \enquote{casc}.

\end{document}
