\documentclass{scrartcl}

\usepackage{amsmath}
\usepackage{amssymb}
\usepackage{oz}
\usepackage{graphicx}
\usepackage{subcaption}
\usepackage{multirow}
\usepackage{listings}
\usepackage{csquotes}
\usepackage{enumerate}
\usepackage{scrextend}

\title{Excerpt of CVC5-gg at the SMT Competition 2021}

\begin{document}
\begin{center}
    \Large{\textbf{Excerpt of}}

    \LARGE{CVC5-gg at the SMT Competition 2021}

    \large{by Clark Barrett, Andres Nötzli, Alex Ozdemir1, Andrew Reynolds,
    Cesare Tinelli, Amalee Wilson, and Haoze Wu}
\end{center}

\vspace{1cm}

\begin{addmargin}[0.2\linewidth]{0.2\linewidth}
    \begin{center}
        \textbf{Key questions}
    \end{center}
    \begin{enumerate}[i]
        \item What is the state of the art in parallel SMT solving?
        \item How successfull are parallel SMT solvers?
        \item Are they used particularly in software verification?
        \item Have (parallel) porfolios been used in SMT solving?
    \end{enumerate}
\end{addmargin}

\vspace{1cm}

cvc5-gg performs a recursive partitioning of the formula
or in other words it is a divide-and-conquer SMT solver.
Each recursive step consists of an attempt to solve formula (using cvc5)
and if it times out, a subsequent split.
The formula \(F\) is transformed into the equisatisfiable formula
\(F \wedge C_1 \vee \dots \vee F \wedge C_d\) where \(C_1, \dotsc, C_d\) are cubes
encoding a partitioning of the search space.
The sub-problems \(F \wedge C_1, \dotsc, F \wedge C_d\) are added to a queue
and will be solved with an increased timeout.

The splitter abuses cvc5 to generate the cubes.
cvc5 is a CDCL(T) solver hence the calls to theory solvers can be intercepted
after a model for the SAT skeleton was found.
The decision literals that cvc5 used to produce that model is used as the cube \(C_1\)
(only the first \(log_2(d)\) literals ar used).
The negation of that cube is added as an assertion to the formula
and another cube is produced in the same way.
It may occur that not enough cubes can be produced.
In that case, lemmas produced by the theory solvers may yield more cubes.
Either way, the final cube will be \(C_n = \neg C_1 \vee \dots \vee \neg C_{n-1}\).

cvc5-gg uses gg for scheduling sub-problems and executing solvers.

\end{document}
