\documentclass{scrartcl}

\usepackage{amsmath}
\usepackage{amssymb}
\usepackage{oz}
\usepackage{graphicx}
\usepackage{subcaption}
\usepackage{multirow}
\usepackage{listings}
\usepackage{csquotes}
\usepackage{enumerate}
\usepackage{scrextend}

\title{Comparing Verification Condition Generation with Symbolic Execution: An Experience Report}

\begin{document}
\begin{center}
    \Large{\textbf{Excerpt of}}

    \LARGE{Comparing Verification Condition Generation
    with Symbolic Execution: An Experience Report}

    \large{by Ioannis T. Kassios, Peter Müller, and Malte Schwerhoff}
\end{center}

\vspace{1cm}

\begin{addmargin}[0.2\linewidth]{0.2\linewidth}
    \begin{center}
        \textbf{Key questions}
    \end{center}
    \begin{enumerate}[i]
        \item Can these two be used as an approach to diversification in a (parallel) portfolio?
        \item Can they both be implemented as part of the same verification tool?
        \item What are \enquote{Cut-Points}?
        \item How would Cut-Points be used as a diversification tactic in a (parallel) portfolio?
    \end{enumerate}
\end{addmargin}

\vspace{1cm}

The paper abstracts automated program verification into three layers:
the specification methodology, the program verifier, and the theorem prover.
The program verifier extracts proof obligations which are passed to the theorem prover.
There are two approaches to design program verifiers:
Verification Condition Generation (VCG) and Symbolic Execution (SE).
VCG \enquote{uses programming calculi such as weakest preconditions
to compute a formula whose validity entails the correctness of the program}.
SE \enquote{executes a program using symbolic instead
of concrete values and accumulates constraints on those values, which are
used to generate proof obligations}.

The paper uses the Chalice language and compares two verification setups:
\begin{enumerate}
    \item Chalice's VCG-based verifier (Z3 as verifier)
    \item a new SE engine (Z3 as verifier)
\end{enumerate}.

\textit{Note: KeY uses SE and not VCG, hence trying to use VCG/SE for diversification entails
using a verification tool other than KeY.}

The paper does not beat around the bush with its conclusion that
\enquote{SE is generally more efficient}, albeit by only a constant factor.
However, they do concede that it \enquote{is not clear to what degree our observations
apply to KeY-style SE}.
The key differences being that KeY uses JML as specification language
and the handling of heap properties.

\textit{Note: This interpretation of their results does not at all lend itself
    to a promise of good diversification through VCG/SE}.

\end{document}
