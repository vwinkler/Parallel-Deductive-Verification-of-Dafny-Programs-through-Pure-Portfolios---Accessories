\documentclass{scrartcl}

\usepackage{biblatex}
\usepackage{csquotes}
\usepackage{draftwatermark}

\addbibresource{sources.bib}

\SetWatermarkScale{5}

\subject{Exposé}
\title{Parallelization of Deductive Software Verification}
\author{
    Vincent Winkler\\
    \vspace{1em}\\
    Reviewer: Prof. Dr. Bernhard Beckert\\
    \vspace{1em}\\
    Advisor: Dr. Mattias Ulbrich
}

\begin{document}

\maketitle

This exposé was produced during the early stages of a master thesis
surrounding the topic of parallelization of software verification.
Its aim is to clarify the goals of the thesis, provide a schedule,
and make sure the work heads in the right direction.
Furthermore, it functions as an agreement between the thesis reviewer and the student
and defines the expectations that will be evaluated in the assessment of the thesis.

This exposé is an instance of a type of document defined in
\enquote{Handreichung zur Erstellung eines Exposés für eine Abschlussarbeit}.

\section{Exposition}
\subsection{Object of study}
The KeY verification tool is the foundation for this research.
The thesis sets out to modify or extend KeY in order to make more use of parallelization
and thus increase runtime performance.
This modification is the primary artifact of the thesis.
It is examined in the light of the question whether it sufficiently improves the runtime
performance to increase KeY's usability.
The artifact includes multiple parallelization approaches.
The thesis seeks to determine how to apply these approaches to maximum effect
and how they compare among each other.
A particular interest is their individual performance impact.

These rather tangible objects of research are expected
to enhance present knowledge on a more abstract question:
\enquote{How to parallelize deductive, formal software verification}.

\subsection{Research approach}
Prior to the work on the thesis,
two approaches for parallelization have been recognized to be promising.
The approach that will actually be followed
or if both will be followed is not yet decided
and depends on the problems that come up during their implementation.
Particulars are given below.

The first approach is the parallelization of KeY itself.
Internally, KeY uses a tree structure, further branching out as it works on its tasks.
As work on one branch is mostly independent from work on another,
a \enquote{by-branch} parallelization is possible.
Implementation of this approach is made more complicated
by datastructures shared between branches.
Thus spotting these datastructures and parallelizing them
before assigning the branches to different threads (e.g. from a thread pool),
is required.
This approach can be followed until KeY's entire feature set
is working in the parallel version
or just until only a sufficiently significant subset is supported.

KeY can also be viewed as a black box that outputs SMT proof obligations.
These proof obligations are already being run concurrently in different SMT solver instances.
Similarly, running a single proof obligation in several instances (\enquote{parallel portfolio})
constitutes a parallelization and the second approach for the thesis.
As with every portfolio, diversification is crucial.
The solver instances should differ in the choice of SMT solver,
in their configuration and/or have different seeds.
An additional dimension to the diversification has been called \enquote{Cut-Points}.
% ToDo: describe what that means
Parallelization of the innards of SMT solvers themselves is not planned to be a task in this thesis,
however the state of the art in parallel SMT solving should be considered.

\subsection{Scientific relevance}
Increasing the runtime performance of software verification tools in general and KeY in particular
has a positive effect on their usability.
This research aims to further knowledge of effective approaches to parallelization
in the field of software verification.
Lastly, examining the effectiveness of approaches to diversification as part of a portfolio
yields information on their effectiveness in general.

\section{Research goal}
On completion there will be a version of the KeY verification tool
(hereby known as \enquote{more parallel version of KeY})
that makes use of at least one of the above-mentioned parallelization approaches.

The artifact and the implemented, individual approaches will be evaluated
against the benchmarks that are part of KeY's nightly builds.
Additionally, if implemented,
the portfolio approach will be evaluated with applicable benchmarks from SMT-LIB.
Runtime and speedup are particularly interesting quantities for this purpose,
hence runtime and speedup plots will be the artifacts of the evaluation.
They'll allow comparing KeY with the more parallel version of KeY and
both parallelization approaches among each other, if both will be implemented.
Moreover, they'll provide the basis for finding an appropriate configuration
and implementation for either approach.
Besides the results, an attempt at their explanation will also be part of the evaluation.

A decrease in runtime by a factor of at least two would be considerable.
Reducing the asymptotic time complexity however, is unobtainable
as the problems are undecidable.

\section{State of the art}
In 2018 Hyvärinen and Wintersteiger \cite{HJW2018} provided an overview of
recent advances in parallel SMT solving.
It states that, next to search-space partitioning and decomposition,
parallel portfolios are an applicable approach towards parallel SMT solving.
Hyvärinen and Wintersteiger particularly reference \cite{WHM2009}, a paper from 2009.

In this paper Wintersteiger, Hamadi and de Moura describe their parallelization of the
SMT solver Z3.
They test three different portfolio approaches using the QF\_IDL
(quantifier-free integer difference logic) benchmarks from SMT-LIB.
Diversification in those three approaches is accomplished through using
different SMT solvers, a diversified SAT solving strategy
and randomized seeds respectively.
Bumping into problems with parallelization overhead due to memory bandwidth,
they did not achieve a sufficient speedup on most benchmarks.
However, selecting only problems that take sequential Z3 at least 60 seconds to solve,
their best portfolio yielded an average speedup of \(3.5\) on four cores.

The work proposed in this exposé sets itself apart from Wintersteiger, Hamadi and de Moura's work
in the approach taken to diversify the solvers in the portfolio.
A benchmark suite more focused on deductive software verification leads to further distinction.

\section{Schedule}
\begin{itemize}
    \item Parallel SMT Portfolio (\(6, wk \le t \le 12\, wk\))
    \begin{enumerate}
        \item read literature on parallel portfolios and plan approach
            (\(1, wk \le t \le 3\, wk\))
        \item implement an evaluation environment
            (\(1\, wk \le t \le 2\, wk\))
        \item implement and evaluate parallel portfolio
            (\(4\, wk \le t \le 7\, wk\))
    \end{enumerate}
    \item Parallel KeY (\(6\, wk \le t \le 12\, wk\))
    \begin{enumerate}
        \item become acquainted with KeY
            (\(1\, wk \le t \le 2\, wk\))
        \item implement an evaluation environment
            (\(2\, wk \le t \le 4\, wk\))
        \item implement and evaluate parallel KeY
            (\(3\, wk \le t \le 6\, wk\))
    \end{enumerate}
    \item write Thesis (\(8\, wk \le t \le 12\, wk\))
\end{itemize}

The implementation of an evaluation environment is finished as soon as
all important plots (at least runtime and speedup) can be generated from the benchmarks.
Within the meaning of this exposé, one is considered acquainted with KeY,
as soon as they know where the parallelization of the branches would have to be
implemented in the code.

Along the \enquote{Parallel SMT Portfolio}-approach the
\enquote{become acquainted with KeY}-step of the \enquote{Parallel KeY}-approach
will be followed.
Then, for further procedure, exactly one of the following options may be taken:
\begin{itemize}
    \item fully implementing the portfolio, abandoning the parallelization of KeY
    \item fully implementing the portfolio, partially implementing the parallelization of KeY
    \item abandoning the portfolio, fully implementing the parallelization of KeY
\end{itemize}

Taking into account that the total time cannot exceed 26 weeks,
the aggregated time spent on both approaches has to be less than 14 weeks
before work on the thesis must begin.
This amounts to a total time \(t_{total}\) with \(22\, wk \le t_{total} \le 26\, wk\).
\printbibliography

\end{document}
