\documentclass{scrartcl}

\usepackage{csquotes}

\subject{Exposé}
\title{Parallelization of Software Verification}
\author{
    Vincent Winkler\\
    \vspace{1em}\\
    Reviewer: Prof. Dr. Bernhard Beckert\\
    \vspace{1em}\\
    Advisor: Dr. Mattias Ulbrich
}

\begin{document}

\maketitle

This exposè was produced during the early stages of a master thesis
surrounding the topic of parallelization of software verification.
Its aim is to clarify the goals of the thesis, provide a schedule,
and make sure the work heads in the right direction.
Furthermore it functions as an agreement between the thesis reviewer and the student
and defines the expectations that will be evaluated in the assessment of the thesis.

This exposè is an instance of a type of document defined in
\enquote{Handreichung zur Erstellung eines Exposés für eine Abschlussarbeit}.

\section{Exposition}
\subsection{Object of study}
The KeY Verification Tool is the foundation for this research.
The thesis sets out to modify or extend KeY in order to make more use of parallelization
and thus increase runtime performance.
This modification is the primary artefact of the thesis.
It is examined in the light of the question whether it sufficeintly improves the runtime
performance to increase KeY's usability.
The artefact includes multiple parallelization approaches.
The thesis seeks to determine how to apply these approaches to maximum effect
and how they compare among each other.
A particular interest is their individual performance impact.

These rather tangible objects of research are expected
to enhance present knowledge on a more abstract question:
\enquote{How to parallelize formal software verification}.

\subsection{Research approach}
Prior to the work on the thesis,
two approaches for parallelization have been recognized to be promising.
One is the parallelization of KeY itself.
Internally, KeY uses a tree structure, further branching out as it works on its tasks.
As work on one branch is mostly independend from work on another,
a \enquote{by-branch} parallelization is possible.
Implementation of this approach is made more complicated
by datastructures shared between branches.
Thus spotting these datastructures and parallelizing them
before assigning the branches to different threads (e.g. from a thread pool),
is required.

KeY can also be viewed as a black box that outputs SMT proof obligations.
These proof obligations are already being run concurrently in different SMT solver instances.
Similarly, it is of interest to run a single proof obligation in several instances
composing a \enquote{parallel portfolio}.
As with every portfolio, diversification is crucial.
The solver instances should differ in the choice of SMT solver,
in their configuration and/or have different seeds.
An additional dimension to the diversification has been called \enquote{Cut-Points}.
% ToDo: describe what that means
Parallel portfolios are the second approach explored in the thesis.
Parallelization of the innards of SMT solvers themself is not planned to be a task in this thesis,
however the state of the art in parallel SMT solving should be considered.

\end{document}
